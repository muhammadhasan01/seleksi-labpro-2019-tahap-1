\documentclass{article}

\usepackage{geometry}
\usepackage{amsmath}
\usepackage{graphicx, eso-pic}
\usepackage{listings}
\usepackage{hyperref}
\usepackage{multicol}
\usepackage{fancyhdr}
\usepackage{mathtools}

\DeclarePairedDelimiter\floor{\lfloor}{\rfloor}

\pagestyle{fancy}
\fancyhf{}
\hypersetup{ colorlinks=true, linkcolor=black, filecolor=magenta, urlcolor=cyan}
\geometry{ a4paper, total={170mm,257mm}, top=20mm, right=20mm, bottom=20mm, left=20mm}
\lhead{Seleksi Tahap 1 Laboratorium Pemrograman 2019}
\setlength{\parindent}{0pt}
\setlength{\parskip}{0.3em}
\renewcommand{\headrulewidth}{0pt}
\rfoot{\thepage}
\lfoot{Seleksi Tahap 1 Labpro 2021}
\lstset{
    basicstyle=\ttfamily\small,
    columns=fixed,
    extendedchars=true,
    breaklines=true,
    tabsize=2,
    prebreak=\raisebox{0ex}[0ex][0ex]{\ensuremath{\hookleftarrow}},
    frame=none,
    showtabs=false,
    showspaces=false,
    showstringspaces=false,
    prebreak={},
    keywordstyle=\color[rgb]{0.627,0.126,0.941},
    commentstyle=\color[rgb]{0.133,0.545,0.133},
    stringstyle=\color[rgb]{01,0,0},
    captionpos=t,
    escapeinside={(\%}{\%)}
}

\begin{document}

\begin{center}
    \section*{Pencari Pasangan - Versi I} % ganti judul soal

    \begin{tabular}{ | c c | }
        \hline
        Batas Waktu  & 1s \\    % jangan lupa ganti time limit
        Batas Memori & 64MB \\  % jangan lupa ganti memory limit
        \hline
    \end{tabular}
\end{center}

\begin{center}
    \textbf{Perbedaan antara versi hanya pada batasan saja.}
\end{center}

\subsection*{Deskripsi}

Diberikan $N$ bilangan bulat, $A_1, A_2, \dots, A_N$, serta bilangan bulat $L$ dan $R$. Carilah banyaknya pasangan bilangan bulat $(i, j)$ $(i < j)$ sehingga $L \leq A_i + A_j \leq R$.

\subsection*{Format Masukan}
\begin{itemize}
\item{Baris Pertama, berisi satu buah bilangan bulat $N$, $L$, dan $R$, masing - masing menyatakan banyaknya bilangan, nilai bilangan bulat $L$ dan nilai bilangan bulat $R$.}

\item{Baris kedua, berisi $N$ bilangan bulat $A_1, \dots, A_N$, menyatakan nilai-nilai dari $N$ bilangan tersebut.}

\end{itemize}

\subsection*{Format Keluaran}
Keluarkan satu baris berisi satu bilangan bulat yang merupakan jawaban dari soal.

\subsection*{Batasan Input}

\begin{itemize}
    \item{$2 \leq N \leq 10^2$}
    \item{$1 \leq A_i \leq 10^9$}
    \item{$1 \leq L_i \leq R_i \leq 10^9$}
\end{itemize}

\begin{multicols}{2}
\subsection*{Contoh Masukan}
\begin{lstlisting}
5 4 7
4 2 5 1 3
\end{lstlisting}
\columnbreak
\subsection*{Contoh Keluaran}
\begin{lstlisting}
7
\end{lstlisting}
\vfill
\null
\end{multicols}

\subsection*{Penjelasan}

Pasangan bilangan bulat $(i, j)$ yang memenuhi adalah:
\begin{itemize}
    \item{(1, 2)}
    \item{(1, 4)}
    \item{(1, 5)}
    \item{(2, 3)}
    \item{(2, 5)}
    \item{(3, 4)}
    \item{(4, 5)}
\end{itemize}

Sehingga terdapat 7 pasangan yang memenuhi.

\end{document}