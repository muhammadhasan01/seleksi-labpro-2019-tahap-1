\documentclass{article}

\usepackage{geometry}
\usepackage{amsmath}
\usepackage{graphicx, eso-pic}
\usepackage{listings}
\usepackage{hyperref}
\usepackage{multicol}
\usepackage{fancyhdr}
\usepackage{mathtools}

\DeclarePairedDelimiter\floor{\lfloor}{\rfloor}

\pagestyle{fancy}
\fancyhf{}
\hypersetup{ colorlinks=true, linkcolor=black, filecolor=magenta, urlcolor=cyan}
\geometry{ a4paper, total={170mm,257mm}, top=20mm, right=20mm, bottom=20mm, left=20mm}
\lhead{Seleksi Tahap 1 Laboratorium Pemrograman 2019}
\setlength{\parindent}{0pt}
\setlength{\parskip}{0.3em}
\renewcommand{\headrulewidth}{0pt}
\rfoot{\thepage}
\lfoot{Seleksi Tahap 1 Labpro 2019}
\lstset{
    basicstyle=\ttfamily\small,
    columns=fixed,
    extendedchars=true,
    breaklines=true,
    tabsize=2,
    prebreak=\raisebox{0ex}[0ex][0ex]{\ensuremath{\hookleftarrow}},
    frame=none,
    showtabs=false,
    showspaces=false,
    showstringspaces=false,
    prebreak={},
    keywordstyle=\color[rgb]{0.627,0.126,0.941},
    commentstyle=\color[rgb]{0.133,0.545,0.133},
    stringstyle=\color[rgb]{01,0,0},
    captionpos=t,
    escapeinside={(\%}{\%)}
}

\begin{document}

\begin{center}
    \section*{Median di Subsegment 1} % ganti judul soal

    \begin{tabular}{ | c c | }
        \hline
        Batas Waktu  & 2s \\    % jangan lupa ganti time limit
        Batas Memori & 256MB \\  % jangan lupa ganti memory limit
        \hline
    \end{tabular}
\end{center}

\begin{center}
    \textbf{Perbedaan antara versi hanya pada batasan saja.}
\end{center}

\subsection*{Deskripsi}

Diberikan $N$ buah bilangan bulat $A_1, A_2, \dots, A_N$ pada sebuah array, beserta $Q$ buah query, dengan setiap query berisi dua bilangan bulat $L$ dan $R$. Untuk setiap query yang diberikan, tentukanlah median dari $A_L, A_{L + 1}, \dots, A_R$. Jika $R - L + 1$ bernilai genap, maka cari nilai urutan (tidak menurun) ke-$(R - L + 1) / 2$.

\subsection*{Format Masukan}
\begin{itemize}
\item{Baris Pertama, berisi satu buah bilangan bulat $N$, menyatakan banyak bilangan pada array.}

\item{Baris kedua, berisi $N$ bilangan bulat $A_1, \dots, A_N$, menyatakan isi dari array.}

\item{Baris ketiga, berisi satu buah bilangan bulat $Q$, menyatakan banyak query yang diberikan.}

\item{$Q$ baris berikutnya, berisi dua bilangan bulat $L_i, R_i$ menyatakan query ke-$i$.}

\end{itemize}

\subsection*{Format Keluaran}
Keluarkan $Q$ baris, pada baris ke-$i$ keluarkan satu bilangan bulat berupa nilai median untuk query ke-$i$.

\subsection*{Batasan Input}

\begin{itemize}
    \item{$1 \leq N \leq 10^4$}
    \item{$1 \leq A_i \leq 10^9$}
    \item{$1 \leq Q \leq 10^2$}
    \item{$1 \leq L_i \leq R_i \leq N$}
\end{itemize}

\begin{multicols}{2}
\subsection*{Contoh Masukan}
\begin{lstlisting}
5
1 4 3 2 9
3
1 3
2 4
4 5
\end{lstlisting}
\columnbreak
\subsection*{Contoh Keluaran}
\begin{lstlisting}
3
3
2
\end{lstlisting}
\vfill
\null
\end{multicols}

\subsection*{Penjelasan}

Untuk query kedua, kita akan dapatkan urutan bilangan menjadi:

$A_4, A_3, A_2 = 2, 3, 4$

Sehingga median yang didapatkan adalah 3.

\end{document}