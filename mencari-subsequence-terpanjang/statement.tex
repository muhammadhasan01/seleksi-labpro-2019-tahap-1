\documentclass{article}

\usepackage{geometry}
\usepackage{amsmath}
\usepackage{graphicx, eso-pic}
\usepackage{listings}
\usepackage{hyperref}
\usepackage{multicol}
\usepackage{fancyhdr}
\usepackage{mathtools}

\DeclarePairedDelimiter\floor{\lfloor}{\rfloor}

\pagestyle{fancy}
\fancyhf{}
\hypersetup{ colorlinks=true, linkcolor=black, filecolor=magenta, urlcolor=cyan}
\geometry{ a4paper, total={170mm,257mm}, top=20mm, right=20mm, bottom=20mm, left=20mm}
\lhead{Seleksi Tahap 1 Laboratorium Pemrograman 2019}
\setlength{\parindent}{0pt}
\setlength{\parskip}{0.3em}
\renewcommand{\headrulewidth}{0pt}
\rfoot{\thepage}
\lfoot{Seleksi Tahap 1 Labpro 2021}
\lstset{
    basicstyle=\ttfamily\small,
    columns=fixed,
    extendedchars=true,
    breaklines=true,
    tabsize=2,
    prebreak=\raisebox{0ex}[0ex][0ex]{\ensuremath{\hookleftarrow}},
    frame=none,
    showtabs=false,
    showspaces=false,
    showstringspaces=false,
    prebreak={},
    keywordstyle=\color[rgb]{0.627,0.126,0.941},
    commentstyle=\color[rgb]{0.133,0.545,0.133},
    stringstyle=\color[rgb]{01,0,0},
    captionpos=t,
    escapeinside={(\%}{\%)}
}

\begin{document}

\begin{center}
    \section*{Mencari yang Terpanjang} % ganti judul soal

    \begin{tabular}{ | c c | }
        \hline
        Batas Waktu  & 1s \\    % jangan lupa ganti time limit
        Batas Memori & 256MB \\  % jangan lupa ganti memory limit
        \hline
    \end{tabular}
\end{center}

\subsection*{Deskripsi}

Diberikan $N$ data, setiap data $(D_i)$ berisi tiga tuple $(X_i, Y_i, Z_i)$. Anda disuruh mencari panjang dari \href{https://en.wikipedia.org/wiki/Subsequence}{subsequence} terpanjang dari $N$ data tersebut, sedemikian sehingga untuk setiap $\left(i < j\right)$ dari subsequence itu berlaku $\left(X_i < X_j, Y_i < Y_j, Z_i < Z_j\right)$.

\subsection*{Format Masukan}

Baris Pertama berisi satu bilangan bulat $N$, menyatakan banyaknya data.

$N$ baris berikutnya, berisi informasi $D_i$ yakni masukkan berupa tiga bilangan bulat $(X_i, Y_i, Z_i)$, menyatakan nilai tuple dari bilangan ke-$i$ $(1 \leq i \leq N)$.


\subsection*{Format Keluaran}
Keluarkan satu baris berisi bilangan bulat berupa jawaban dari soal.

\subsection*{Batasan Input}

\begin{itemize}
    \item{$2 \leq N \leq 10^3$}
    \item{$1 \leq X_i, Y_i, Z_i \leq 10^3$}
\end{itemize}

\begin{multicols}{2}
\subsection*{Contoh Masukan}
\begin{lstlisting}
5
4 5 2
1 5 7
2 6 8
5 6 3
6 7 4

\end{lstlisting}
\columnbreak
\subsection*{Contoh Keluaran}
\begin{lstlisting}
3
\end{lstlisting}
\vfill
\null
\end{multicols}

\subsection*{Penjelasan}


Salah satu subsequence terpanjang yang memenuhi adalah $D_1, D_4, D_5$.

\end{document}