\documentclass{article}

\usepackage{geometry}
\usepackage{amsmath}
\usepackage{graphicx, eso-pic}
\usepackage{listings}
\usepackage{hyperref}
\usepackage{multicol}
\usepackage{fancyhdr}
\usepackage{mathtools}

\DeclarePairedDelimiter\floor{\lfloor}{\rfloor}

\pagestyle{fancy}
\fancyhf{}
\hypersetup{ colorlinks=true, linkcolor=black, filecolor=magenta, urlcolor=cyan}
\geometry{ a4paper, total={170mm,257mm}, top=20mm, right=20mm, bottom=20mm, left=20mm}
\lhead{Seleksi Tahap 1 Laboratorium Pemrograman 2019}
\setlength{\parindent}{0pt}
\setlength{\parskip}{0.3em}
\renewcommand{\headrulewidth}{0pt}
\rfoot{\thepage}
\lfoot{Seleksi Tahap 1 Labpro 2019}
\lstset{
    basicstyle=\ttfamily\small,
    columns=fixed,
    extendedchars=true,
    breaklines=true,
    tabsize=2,
    prebreak=\raisebox{0ex}[0ex][0ex]{\ensuremath{\hookleftarrow}},
    frame=none,
    showtabs=false,
    showspaces=false,
    showstringspaces=false,
    prebreak={},
    keywordstyle=\color[rgb]{0.627,0.126,0.941},
    commentstyle=\color[rgb]{0.133,0.545,0.133},
    stringstyle=\color[rgb]{01,0,0},
    captionpos=t,
    escapeinside={(\%}{\%)}
}

\begin{document}

\begin{center}
    \section*{Banyak Domain 1} % ganti judul soal

    \begin{tabular}{ | c c | }
        \hline
        Batas Waktu  & 1s \\    % jangan lupa ganti time limit
        Batas Memori & 256MB \\  % jangan lupa ganti memory limit
        \hline
    \end{tabular}
\end{center}

\begin{center}
    \textbf{Perbedaan antara versi hanya pada batasan saja.}
\end{center}

\subsection*{Deskripsi}

Diberikan $N$ buah domain/area, $(L_1, R_1), (L_2, R_2), …, (L_N, R_N)$, berupa representasi range dari domain. Dan diberikan $Q$ query dengan pada setiap query diberikan sebuah bilangan $A$.
Pada setiap query, hitunglah banyak domain dimana bilangan tersebut berada.

Bilangan $A$ akan berada di suatu domain $(L, R)$ jika dan hanya jika dipenuhi $L \leq A \leq R$.

\subsection*{Format Masukan}
\begin{itemize}
\item{Baris Pertama, berisi satu bilangan bulat $N$, menyatakan banyak domain yang ada.}

\item{$N$ baris berikutnya, pada masing-masing baris berisi domain ke-$i$ yang terdiri dari dua bilangan bulat $L_i$ dan $R_i$.}

\item{Baris berikutnya, berisi satu bilangan bulat $Q$, menyatakan banyak query.}

\item{$Q$ baris berikutnya, pada masing-masing baris berisi query ke-$i$ berupa satu bilangan bulat $A_i$.}

\end{itemize}

\subsection*{Format Keluaran}
Keluarkan $Q$ baris, pada baris ke-$i$ terdapat satu bilangan bulat berupa banyak domain dimana query ke-$i$ berada.

\subsection*{Batasan Input}

\begin{itemize}
    \item{$1 \leq N \leq 10^3$}
    \item{$1 \leq L_i \leq R_i \leq 10^5$}
    \item{$1 \leq Q \leq 10^3$}
    \item{$1 \leq A_i \leq 10^5$}
\end{itemize}

\begin{multicols}{2}
\subsection*{Contoh Masukan}
\begin{lstlisting}
2
1 10
6 20
3
2
7
19
\end{lstlisting}
\columnbreak
\subsection*{Contoh Keluaran}
\begin{lstlisting}
1
2
1
\end{lstlisting}
\vfill
\null
\end{multicols}

\end{document}